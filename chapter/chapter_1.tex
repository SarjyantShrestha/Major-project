     \chapter{Introduction}
        \pagenumbering{arabic}
        \section{Background Introduction}
        NLP is a branch of linguistics, computer science, and artificial intelligence concerned with computer human interaction, mainly how to design computers to process and evaluate huge volumes of natural language data \cite{asudani2023impact}. Word embedding is a fundamental concept in natural language processing (NLP). Word embedding is a real-valued vector representation of words by embedding both semantic and syntactic meanings obtained from unlabeled large corpus \cite{Wang_Wang_Chen_Wang_Kuo_2019}. It is of n-dimensional distributed representation of a text that attempts to capture the meanings of the words \cite{asudani2023impact}. Word embeddings can be obtained using language modeling and feature learning techniques, where words or phrases from the vocabulary are mapped to vectors of real numbers \cite{enwiki:1219561882}. Pre-trained word embeddings encode general word semantics and lexical regularities of natural language, and have proven useful across many NLP tasks, including word sense disambiguation, machine translation, and sentiment analysis, to name a few \cite{moreo2019wordclass}. \\ 
        Nepali is one of the languages that uses Devanagari, a script used in many languages spoken in Asian countries. It has been rarely used for Natural Language Processing services. Due to its complex grammatical structure and rich characters extracting fruitful information from the corpus has been challenging\cite{NepaliBERT}.
        Nepali is the official language of Nepal. It is spoken by more than 20 million people, mainly in Nepal, and other places in the world including Bhutan, India and Myanmar \cite{niraula2020linguistic}. Nepali can be quite complex due to its many sounds, grammar rules, and ways to change words.

        In today's rapidly evolving technological landscape, AI is on the rise, due to this, the development of NLP technologies has reached to extreme heights and is still growing every single day. NLP enables computers to understand and generate human language, this new tech has led to the creation of highly useful general-purpose technologies (GPT), including chatbots and virtual assistants. NLP has seen extensive development, particularly in English, but it is crucial to recognize that the benefits of such technologies are not exclusive to English speakers only. Just as the complexity of the Nepali language poses unique challenges, the creation and advancement of NLP technologies adapted to individual languages, like Nepali, hold immense potential for empowering communities and enhancing accessibility to digital resources for Nepali language. By filling the gap between technological innovation and linguistic diversity, we can unlock new possibilities for communication, education, and cultural preservation across the globe. \\
        In recent years, a few word embedding models based on the Nepali language have been developing. Some of them include NPVec1 \cite{koirala-niraula-2021-npvec1}, NepBERTa \cite{timilsina-etal-2022-nepberta}, and Word Embedding in Nepali Language using Word2Vec \cite{inproceedings}. 

        \section{Problem Statement} 
        Even though Word Embeddings can be directly learned from raw texts in an unsupervised fashion, gathering a large amount of data for its training remains a huge challenge in itself for a low-resource language such as Nepali \cite{koirala-niraula-2021-npvec1}.
        \section{Objective}
            The main aim of this project is:
            \begin{itemize}
                \item To identify manipulated digital media content, particularly facial features and images.
                \item To implement cutting-edge deep learning and machine learning techniques.
            \end{itemize}

