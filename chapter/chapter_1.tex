     \chapter{Introduction}
        \pagenumbering{arabic}
        \section{Background Introduction}
        In recent years, the landscape of digital image manipulation has changed with the advent of deepfake technology.
        This innovative approach, based on deep learning techniques, has attracted significant attention as a means of generating images by seamlessly replacing facial features of one person with those of another.
        Titled "deepfakes" by a Reddit user in 2017, these operations often utilize advanced adversarial models such as generative adversarial networks (GANs).

        In particular, the technology has been controversially used to overlay celebrity faces onto explicit content, raising concerns about fake pornography, misinformation, financial fraud and misinformation.
        Despite the ethical challenges associated with deepfakes, it is important to recognize their positive applications in areas such as virtual reality, film editing, and production.

        The basic working principle behind deepfake involves the complex process of combining, replacing, merging,  and overlaying images.
        These operations, using deep learning and machine learning techniques, force changes to digital images and videos, highlighting both the potential benefits and ethical considerations associated with this rapidly evolving technology.
        \section{Problem Statement}
        In the rapidly evolving landscape of computing and automation technologies, the scope of possibilities continues to expand.
        Artificial intelligence (AI) is at the heart of driving unprecedented advances in areas such as predictive analytics, weather forecasting, automation, and the creation of advanced entities such as deepfakes containing AI-generated video, audio, and images.
        While these technological advances are undoubtedly transformative, the misuse and abuse of such capabilities raise serious concerns.

        Lately, there has been an increase in the creation of deepfakes that manipulate the faces of celebrities and ordinary people using just a single image and advanced deep-learning algorithms.
        This issue is becoming increasingly important as potentially harmful and illegal images of victims are made public.
        This rise in false activity not only threatens individual privacy but also has broad implications for public trust and safety.
        As the credibility of deepfakes increases, the potential for malicious use, misinformation, and reputational damage increases.
        It is important to address this issue head-on by developing advanced detection mechanisms to protect against the harmful consequences of manipulated images.
        
        This proposal aims to contribute to ongoing efforts to reduce the risks associated with deepfakes and enhance the integrity of visual content in the era of advanced AI technologies.
       \section{Objective}
            The main aim of this project is:
            \begin{itemize}
                \item To identify manipulated digital media content, particularly facial features and images.
                \item To implement cutting-edge deep learning and machine learning techniques.
            \end{itemize}