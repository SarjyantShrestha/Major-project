\pagenumbering{roman}

% \vspace*{1in}

% \begin{center}
% 	\large{\textbf{CERTIFICATE}}
% \end{center}
% \vspace{1cm}
% This is to certify that this major project work entitled \textbf{``DeepFake Image Detection"} submitted by Manish Pyakurel (KCE077BCT020), Rupak Neupane (KCE077BCT028), Sarjyant Shrestha (KCE077BCT033) and Srijan Gyawali\\(KCE077BCT036) has been examined and accepted as the partial fulfillment of the requirements for degree of Bachelor in Computer Engineering.\\
% \vspace{1in}
% \begin{multicols}{2}
% 	\begin{center}
% 		..........................................\\
% 		\textbf{Er. Himal Chand Thapa}\\
% 		Senior Lecturer\\
% 		External Examiner\\
% 		Himalaya College of Engineering\\
% 		\textbf{}\\
		
% 	\end{center}

%   \columnbreak
% 	\begin{center}
% 		..........................................\\
% 		\textbf{Er. Subhadra Joshi}\\
%     Lecturer\\
% 		Project Supervisor\\
%     Dept. of Computer Engineering\\
%     Khwopa College of Engineering
% 	\end{center}
% \end{multicols}
% \vspace{1in}
% \begin{center}
% 	..........................................\\
% 	\textbf{Er. Dinesh Gothe}\\
% 	Head of Department\\
% 	Department of Computer Engineering\\
% 	Khwopa College of Engineering
% \end{center}
% \pagebreak



% \large
% 	\chapter*{Copyright}
% \normalsize
% \addcontentsline{toc}{section}{Copyright}
% 	The author has agreed that the library, Khwopa College of Engineering  may make this report freely available for inspection. Moreover, the author has agreed that permission for the extensive copying of this project report for scholarly purpose may be granted by supervisor who supervised the project work recorded here in or, in absence the Head of The Department where in the project report was done. It is understood that the recognition will be given to the author of the report and to Department of Computer Engineering, KhCE in any use of the material of this project report. Copying or publication or other use of this report for financial gain without approval of the department and author’s written permission is prohibited. Request for the permission to copy or to make any other use of material in this report in whole or in part should be addressed to: \\
% 	\vspace{1cm} \\
% 	Head of Department \\
% 	Department of Computer Engineering\\
% 	Khwopa College of Engineering\\
% 	Libali,\\
% 	Bhaktapur, Nepal\\
% \pagebreak


% \large
% \chapter*{Acknowledgement}
% \normalsize
% \addcontentsline{toc}{section}{Acknowledgement}
% We extend our heartfelt gratitude to our Head of Department, Er. Dinesh Gothe, for his invaluable guidance, insightful advice, and unwavering encouragement throughout our Bachelor's program. His motivating suggestions have greatly enriched this project.\\\\
% Additionally, we express our appreciation to Er. Subhadra Joshi and Er. Niranjan Bekoju for their valuable suggestions and continuous support throughout the development of this project.
% \begin{table}[h]
% 	\begin{tabular}{@{}ll}
% 		Manish Pyakurel    & KCE077BCT020 \\
% 		Rupak Neupane      & KCE077BCT028 \\
% 		Sarjyant Shrestha  & KCE077BCT033 \\
% 		Srijan Gyawali     & KCE077BCT036 \\
% 	\end{tabular}
% \end{table}
% \pagebreak

\large
\chapter*{Abstract}
\addcontentsline{toc}{section}{Abstract}
The rise of deepfake technology, powered by advancements in deep learning, has brought a new phase to digital image manipulation. This project addresses the pressing need for a robust deepfake detection system, focusing on facial features, and employs a state-of-the-art ResNet architecture. The primary goal is to develop a deep learning model capable of accurately identifying manipulated facial images. The existing methodologies regarding deepfake technologies and detection have been thoroughly analyzed. For this project, 15k Real and Fake Images from the Deepfake Detection and Reconstruction Challenge were used as the base dataset and augmented to a total of 50k images to train the model. The integration of CNN into the deepfake detection process was chosen, with a specific emphasis on the training and rigorous testing of the model’s efficiency. An accuracy of 93.8\% has been achieved, surpassing all other proposed models in the challenge.\\\\
\textbf{Keywords}: \textit{Deepfake technology, Deep learning, Digital image manipulation, CNN}
\pagebreak

\tableofcontents
% \listoffigures
% \addcontentsline{toc}{section}{List of Figures}
% \listoftables
% \addcontentsline{toc}{section}{List of Tables}
% \pagebreak

% \Large
% \begingroup
% \let\clearpage\relax
% \chapter*{List of Abbreviation}
% \endgroup
% \normalsize
% \addcontentsline{toc}{section}{List of Abbreviations}

% \begin{table}[h]
% 	\begin{tabular}{l l}
% 		\textbf{Abbreviations} & \textbf{Meaning}                                   \\
% 		AI                     & Aritificial Intelligence\\
% 		NLP                    & Natural Language Processing\\
% 		NLTK                   & Natural Language ToolKit\\
% 		XLM                    & Cross-lingual Language Model\\
% 		GAN                    & Generative Adversarial Network                     \\
% 		HFM                    & Handcrafted Face Manipulation                       \\
% 		ICIAP                  & International Conference on Image Analysis and Processing \\
% 		MCNet                  & Multi Level Correction Network                     \\
% 		RaFT                   & Radboud Faces Database                             \\
% 		ROC                    & Receiver Operating Characteristic                   \\
% 		SFFM                   & Shallow-FakeFaceNet                                 \\
% 		StarGANs               & Star Generative Adversarial Networks               \\
% 		StyleGANs              & Style Generative Adversarial Networks               \\
% 	\end{tabular}
% \end{table}
% \pagebreak

%  \textsanskrit{नमस्ते}
%  \textenglish{torpe}