\chapter*{Abstract}
\addcontentsline{toc}{chapter}{Abstract}
The rise of deepfake technology, powered by advancements in deep learning, has brought a new phase to digital image manipulation. This project addresses the pressing need for a robust deepfake detection system, focusing on facial features, and employs state-of-the-art ResNet architecture. The primary goal is to develop a deep learning model capable of accurately identifying manipulated digital media content. We have analysed the existing methodologies regarding deepfake technologies and detection. Data collection involves transforming authentic facial images into deepfakes using FaceSwap. We have opted to integrate ResNet architecture into the deepfake detection process, with a specific emphasis on the training and rigorous testing of the model's efficiency. This project aims to contribute to ongoing efforts in mitigating the risks associated with deepfakes, ensuring the integrity of visual content in the era of advanced AI technologies. The proposed ResNet-based deepfake detection model holds promise for enhancing accuracy and efficiency, addressing challenges posed by manipulated images in contemporary digital environments.
