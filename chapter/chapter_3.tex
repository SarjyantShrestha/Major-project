      \chapter{Requirement Analysis}
        \section{SOFTWARE REQUIREMENT}
            This project requires following softwares:
            \subsection*{Python} 
            Python is our primary language for this project, chosen for its versatility and simplicity. All machine learning frameworks are imported using
            Python, leveraging its dominant position in data science and machine learning. This allows us to seamlessly integrate powerful libraries like TensorFlow and PyTorch for efficient model development.
        
            \subsection*{Pytorch}
            Pytorch is a an open-source machine learning framework. It provides a flexible and dynamic computational graph, which allows for easy experimentation and rapid development of deep learning models. 
                
        \section{FUNCTIONAL REQUIREMENT}
            These are specifications that describe the fundamental capabilities and behaviors a system or product must exhibit to meet the users' needs and achieve its intended purpose. 

            \subsection{Dataset Labeler}
                Dataset labeler is the labelling system that is used to annotate the images as “Real”, or “Fake”.

            \subsection{User-Friendly Dashboard}
                A user-friendly dashboard that allows user to monitor and manage the deepfake detection process.

        \section{NON-FUNCTIONAL REQUIREMENT}
            These are the characteristics and qualities that describe how a system should behave and perform,

            \subsection{Reliability}
                The system aims for high reliability, ensuring accurate predictions with a confidence level of 97 percent.
            \subsection{Maintainability}
                The model is designed with maintainability in mind, allowing for easy updates and further training. This ensures adaptability to evolving datasets and changing input patterns, contributing to sustained efficiency and relevance over time.
            \subsection{Portability}
                The use of standardized and widely supported libraries ensures that the model can seamlessly run on various operating systems and integrate with diverse hardware configurations.
            
        \section{FEASIBILITY STUDY}
            The following points describes the feasibility of the project.

            \subsection{Economic Feasibility}
                The total expenditure of the project is just computational power. The computational resources can be fulfilled with the help of college. Therefore, the project is economically feasible.

            \subsection{Technical Feasibility}
                Large number of already labelled datasets are easily available on the internet which is the most crucial requirement for this project. And with all the resources we have access to, this project is technically feasible.

            \subsection{Operational Feasibility}
                The operational processes, including data labeling and model training, are well-defined and can be efficiently carried out by the project team. Additionally, the project aligns with the existing technical infrastructure and capabilities, making it operationally feasible.

            
