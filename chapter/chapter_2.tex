\chapter{Literature Review}
Deepfakes, which involve sharing facial images without permission, are often carried out without the knowledge or consent of individuals such as celebrities or politicians.
 Notably, historical examples such as the facial changes in  Abraham Lincoln's photographs (Badale et al.
, 2018)\cite{badale2018deep} highlight the long-standing nature of this challenge.
 To address these concerns, Yang, Li, \& Lyu (2019)\cite{yang2019exposing} exploit head pose discrepancies to detect deepfakes and create synthetic faces of different people while preserving the original facial expressions.
 We have proposed a model that can be created.
 Jagdale \& Shah (2019)\cite{jagdale2019novel} introduced his NA-VSR algorithm for super-resolution.
 This includes video conversion to frames, median filtering to remove noise, and bicubic interpolation to improve pixel density.
 Additionally, Yadav \& Salmani (2019)\cite{yadav2019deepfake} explained the working principle of deepfake technology and highlighted the value of high accuracy when exchanging facial images.
 Generative adversarial neural networks (GANs) play a central role in deepfake generation and consist of generators and discriminators.
 The generator synthesizes a fake image from the given dataset, and the discriminator evaluates the authenticity of the generated image.
 The risks inherent in deepfakes, including defamation of character, potential harm to individuals, and the spread of fake news throughout society, highlight the importance of addressing these challenges.
 
 Existing approaches suffer from problems such as inefficiency in deepfake image detection, high error rates, long computation times, and data access inaccuracies.
 Our work focuses on improving efficiency through a different approach to using ResNet architecture for deepfake detection.
 Figure 1 shows a compilation of related studies on deepfake detection (Heidari et al., 2023)\cite{heidari2023deepfake}.

\begin{table}[ht]
  \centering
  \caption{Compilation of related work}
  \small
  \begin{tabular}{|p{2cm}|p{3cm}|p{2.5cm}|p{3cm}|p{3cm}|}
    \hline
    \textbf{Researcher} & \textbf{Contributions} & \textbf{Scope} & \textbf{Advantage} & \textbf{Weakness} \\
    \hline
    L. Verdoliva (2020) & Presenting an overview of contemporary manipulation techniques & Fake media & Deepfake's backstory is presented. Issues and possible solutions are explored. & There has not been any in-depth review of the articles. \\
    \hline
    Tolosana et al. (2020) & Examining face-altering techniques & Image deepfake detection & Different criteria for evaluating articles are taken into account. & It is unclear how articles are chosen for review. \\
    \hline
    Mirsky and Lee (2021) & Providing deepfake creation and detection services & Deepfake in general & Challenges and potential guidance are discussed. & It is unclear how articles are chosen for review. \\
    \hline
    Castillo Camacho and Wang (2021) & Examining DL-based image forensic methods & Image forensic & Taking into account all aspects of the criteria for image forensics. & It is unclear how articles are chosen for review. \\
    \hline
    P. Yu et al. (2021) & Focusing on deepfake video detection, its history, current research, and plans & Deepfake video & An in-depth description of future work. In-depth examination of datasets. & There is no comparison between the articles. \\
    \hline
    Rana et al. (2022) & Demonstrating several cutting-edge deepfake algorithms & DL-ML and statistical models & There is a comparison between the articles. & There is no discussion of all kinds of deepfake applications. \\
    \hline
    Heidari et al. (2023) & Providing a comprehensive review of the literature on deepfake detection techniques based on DL-based algorithms & DL-ML methods in the video, image, audio, and hybrid multimedia detection & An in-depth description of future work. In-depth examination of datasets. Challenges and potential guidance are discussed. & Papers published before 2018 are not allowed.  \\
    \hline
  \end{tabular}
\end{table}

    % L. Verdoliva (2020) & Presenting an overview of contemporary manipulation techniques & Fake media & Deepfake's backstory is presented. Issues and possible solutions are explored. There has not been any in-depth review of the articles. & \\
    % \hline
    % Tolosana et al. (2020) & Examining face-altering techniques & Image deepfake detection & Different criteria for evaluating articles are taken into account. It is unclear how articles are chosen for review. & \\
    % \hline
    % Mirsky and Lee (2021) & Providing deepfake creation and detection services & Deepfake in general & Challenges and potential guidance are discussed. It is unclear how articles are chosen for review. & \\
    % \hline
    % Castillo Camacho and Wang (2021) & Examining DL-based image forensic methods & Image forensic & Taking into account all aspects of the criteria for image forensics. It is unclear how articles are chosen for review. & \\
    % \hline
    % P. Yu et al. (2021) & Focusing on deepfake video detection, its history, current research, and plans & Deepfake video & An in-depth description of future work. In-depth examination of datasets. There is no comparison between the articles. & \\
    % \hline
    % Rana et al. (2022) & Demonstrating several cutting-edge deepfake algorithms & DL-ML and statistical models & There is a comparison between the articles. There is no discussion of all kinds of deepfake applications. & \\
    % \hline
    % Ours & Providing a comprehensive review of the literature on deepfake detection techniques based on DL-based algorithms & DL-ML methods in video, image, audio, and hybrid multimedia detection & An in-depth description of future work. In-depth examination of datasets. Challenges and potential guidance are discussed. Papers published before 2018 are not allowed. & \\
    % \hline
