%\newgeometry{top=0.65in}
\pagenumbering{roman}
		%\addcontentsline{toc}{section}{Copyright}
		%\large
			%\chapter*{Copyright}
		%\normalsize
		%\justify
		%The author has agreed that the library, Khwopa College of Engineeringmay make this report freely available for inspection. Moreover, the authorhas agreed that permission for the extensive copying of this project report for scholarlypurpose may be granted by supervisor who supervised the project work recorded hereinor, in absence the Head of The Department wherein the project report was done. It isunderstood that the recognition will be given to the author of the report and toDepartment of Computer Engineering, KhCE iany use of the material of this project report. Copying or publication or other use of thisreport for financial gain without approval of the department and author’s writtenpermission is prohibited. Request for the permission to copy or to make any other useof material in this report in whole or in part should be addressed to:Head of Department\\\vspace{1.5cm}Department of Computer Engineering\\Khwopa College of Engineering(KhCE)\\Liwali,\\Bhaktapur, Nepal.\\\break

	
	   % \addcontentsline{toc}{section}{Acknowledgement}
		%\large
			%\chapter*{Acknowledgement}
		%\normalsize
		% We would like to thank \textbf{Er. Anish Baral} for his wise counsel, inspiring ideas, and invaluable direction, help, and support throughout this project. We also owe a debt of gratitude to \textbf{Er. Dinesh Gothe} and \textbf{Er. Mukesh Kumar Pokharel} \textbf for their tireless efforts at every stage of the project, as well as for their insightful counsel and recommendations.\\ We are truly grateful for the wisdom and support they have provided us and it isthrough their efforts that we have been able to bring this project to fruition. Onbehalf of the entire team, we express our sincere thanks and appreciation for their invaluable contributions.\\
            


        %\begin{tabular}{p{1.5in}p{3in}}
			%Bijay Kadariya & KCE076BCT013\\
			%Prajwal Acharya & KCE076BCT025\\
			%Prajwal Shrestha & KCE076BCT026\\
			%Sameer Shrestha & KCE076BCT037\\
		%\end{tabular}
		%\break
		
		
		\addcontentsline{toc}{section}{Abstract}
		\large
			\chapter*{Abstract}
		\normalsize
        \noindent
		In recent years, Natural Language Processing (NLP) has made significant strides, particularly in the development of word embeddings that capture both semantic and syntactic meanings of words. This proposal focuses on creating word embeddings for the Nepali language, which remains underrepresented in the realm of NLP due to its complex grammatical structure and rich character set. Despite the progress in NLP, low-resource languages like Nepali face challenges in data collection and model training. This study aims to address these challenges by leveraging pre-trained models and fine-tuning them with a substantial Nepali corpus. The proposed system will utilize transformer-based models, such as BERT, to generate contextualized word embeddings that can be applied to various NLP tasks, including sentiment analysis, machine translation, and question answering. By advancing NLP technologies for the Nepali language, we aim to enhance digital accessibility and empower communities through improved communication and educational tools.

		\noindent
		\textbf{Keywords}: 
		\textit{Word Embeddings, Nepali Language, Natural Language Processing, BERT, Transformer Models, Contextualized Embeddings, Low-Resource Languages\\
 }\\

		\break


	    \tableofcontents

		\addcontentsline{toc}{section}{List of Tables}
		\listoftables
		\break
		\pagebreak

		\addcontentsline{toc}{section}{List of Figures}
		\listoffigures
		\break
	
	
	
		\addcontentsline{toc}{section}{List of Symbols and Abbreviation}
		\Large
			\begingroup
				\let\clearpage\relax
				\chapter*{List of Symbols and Abbreviation}
			\endgroup
   
		\normalsize
		\begin{tabular}{p{1in}p{5in}} 
                AI & Artificial Intelligence\\
                
            \end{tabular}





		\break
		\pagebreak
		
	